\documentclass{article}
\usepackage{graphicx} % Required for inserting images

\usepackage{amsmath}
\usepackage{amsfonts}
\usepackage{hyperref}
\usepackage{algorithm}
\usepackage[noend]{algpseudocode}

\usepackage{tikz}
\usepackage[margin=2.3cm]{geometry}

\usetikzlibrary{positioning}

\title{Computational Complexity -- Homework 3}
\author{Dario Halilovic\and
Antoni Jubés Monforte\and
Marcin Wojnarowski}

\date{December 2024}

\newcommand{\PSPACE}[0]{\mathsf{PSPACE}}
\renewcommand{\P}[0]{\mathsf{P}}
\newcommand{\NP}[0]{\mathsf{NP}}
\newcommand{\SAT}[0]{\textsc{SAT}}

\begin{document}

\maketitle

\section*{Problem 1}

\subsection*{"$\le$"}

Let $T_f$ be the DT computing $f$ with height $D(f)$ and let $T_g$ be the DT computing $g$ with height $D(g)$. We construct $T$ which computes $f \circ g$ by inlining $T_g$ multiple times. Namely, $T$ is constructed from $T_f$: when $T_f$ would query its $i$-th input bit $x_i$, we inline $T_g$ to compute the answer.

The height of $T$ is precisely $D(f) \cdot D(g)$, thus $D(f \circ g) \le D(f) \cdot D(g)$.

\subsection*{"$\ge$"}

From exercise sheet VIII exercise 3, we know that for any lower bound on $D(\cdot)$ there exists an adversarial strategy. More importantly, for a lower bound which is an equality an adversarial strategy also exists. Thus suppose we are given adversarial strategies $s_f$ and $s_g$ for respectively $f$ and $g$ that give us the greatest lower bound (equality) of $D(f)$ and $D(g)$ respectively. We construct an adversarial strategy for $f \circ g$.

We use $n$ instances of $s_g$ and one of $s_f$. Suppose a DT of $f \circ g$ wants to query some bit $x_i^j$ for $i \le m$ and $j \le n$. We use the $j$-th instance of $s_g$ to answer the query about its $i$-th bit. These strategy instances are paused and resumed each time they are used. This way, for each $j \le n$ the value of $g(x^j)$ is still undetermined after $D(g)-1$ queries to $x^j$ bits. On the $D(g)$-th query to $x^j$ we switch to using the $s_f$ strategy for the $j$-th bit.

This yields $D(f \circ g) \ge (D(g) - 1) \cdot n + D(f)$. Notice however that $D(h)$ for any $h$ is smaller or equal to the input size (as this means all input bits have been queried). Thus $D(f) \le n$. This gives us $D(f \circ g) \ge (D(g) - 1) \cdot n + D(f) \ge (D(g) - 1) \cdot D(f) + D(f) = D(g) \cdot D(f)$.

\section*{Problem 2}

\section*{Problem 3}

\end{document}
