\documentclass{article}
\usepackage{graphicx} % Required for inserting images

\usepackage{amsmath}
\usepackage{amsfonts}
\usepackage{hyperref}
\usepackage{algorithm}
\usepackage[noend]{algpseudocode}

\usepackage{tikz}
\usepackage[margin=2.3cm]{geometry}

\usetikzlibrary{positioning}

\title{Computational Complexity -- Homework 3}
\author{Dario Halilovic\and
Antoni Jubés Monforte\and
Marcin Wojnarowski}

\date{December 2024}

\newcommand{\PSPACE}[0]{\mathsf{PSPACE}}
\renewcommand{\P}[0]{\mathsf{P}}
\newcommand{\NP}[0]{\mathsf{NP}}
\newcommand{\SAT}[0]{\textsc{SAT}}

\begin{document}

\maketitle

\section*{Problem 1}

\section*{Problem 2}

We first show that no assignment $x^{1}, \ldots, x^{n}$
satisfies formula $F$. Consider the sequence $y_0 = 1, y_1 = x^{y_0}, y_2
= x^{y_1}, \ldots, y_{n} = x^{y_{n-1}}$ of $n+1$ nodes which is the sequence of
nodes reached from $1$ by following pointers. By constraints 1 and 2, the
sequence of $y_i$ is monotonically increasing. Since $x^1 > 1$, one has $y_1
> y_0$, and therefore by the third constraint, $y_1$ does not point forward to
a self-loop. Generalizing the argument inductively, every $y_i$ does not point
forward to a self-loop, hence the sequence is strictly monotonic. But there are
$n+1$ nodes, hence by the pigeonhole principle there exists $k \in [n]$ such
that $y_i = y_j = k$ for some $i \neq j$, contradicting strict monotonicity.

We now show that any tree-like refutation proof of $F$ requires $\Omega(n)$
depth. Consider the decision tree for the search problem of $F$. We show by an
adversarial argument that any $n - 1$ queries are not enough to find a clause
$C$ such that $C$ is a violation of any of the 3 constraints. In other words, we
show that $D(\textrm{Search}(F)) \geq n - 1 = \Omega(n)$.

Suppose that the decision tree makes at most $n - 1$ queries, and assume for the
sake of simplicity that one query is enough to determine all $\log(n)$ bits of
$x^j$ (it might not be the case in practice, but the following result would
still apply). For any query $x^{j}$, the adversary answers
\[
x^j = \begin{cases}
	\min(\{k > j : x^k \textrm{ is undefined} \lor x^k \neq k\}) & \textrm{if
	} j \textrm{ is reachable from 1 by following pointers}, \\
	j & \textrm{otherwise}
\end{cases}
\]
Because at most $n - 1$ queries are made, there exists some $i \in [n]$ such
that $x^i$ was not queried. Notice that by construction, every constraint is
satisfied, since we do not assign forward pointers to self-loops. Therefore, the
choice of $x^i$ determines which clause is unsatisfied.

If $i = 1$, the constructed assignment only contains self-loops, since $1$ was
never queried. Therefore, the adversary can make either the first or third
constraint fail, by setting $x^1 = 1$ or $x^1 = 2$, respectively.

If $i > 1$, the adversary can choose $x^i$ such that either constraint 2 or
3 fails, by setting $x^i = i - 1$ or $x^i = i$, respectively, since $x^i$ is
reachable from $1$.

In both cases, for any clause that the decision tree outputs, there exists an
adversarial input such that the clause is valid, thereby making the output
incorrect.

\section*{Problem 3}

\end{document}
