\documentclass{article}
\usepackage{graphicx} % Required for inserting images

\usepackage{amsmath}
\usepackage{amsfonts}
\usepackage{hyperref}
\usepackage{algorithm}
\usepackage[noend]{algpseudocode}

\usepackage{tikz}
\usepackage[margin=2.3cm]{geometry}

\usetikzlibrary{positioning}

\title{Computational Complexity -- Homework 3}
\author{Dario Halilovic\and
Antoni Jubés Monforte\and
Marcin Wojnarowski}

\date{December 2024}

\newcommand{\PSPACE}[0]{\mathsf{PSPACE}}
\renewcommand{\P}[0]{\mathsf{P}}
\newcommand{\NP}[0]{\mathsf{NP}}
\newcommand{\SAT}[0]{\textsc{SAT}}

\begin{document}

\maketitle

\section*{Problem 1}

\subsection*{"$\le$"}

Let $t_f$ be the DT computing $f$ with height $D(f)$ and let $t_g$ be the DT computing $g$ with height $D(g)$. We construct $t$ which computes $f \circ g$ by inlining $t_g$ multiple times. Namely, $t$ is constructed from $t_f$: when $t_f$ would query its i-th input bit $x_i$, we inline $t_g$ to compute the answer.

The height of $t$ is precisely $D(f) \cdot D(g)$, thus $D(f \circ g) \le D(f) \cdot D(g)$.


\section*{Problem 2}

\section*{Problem 3}

\end{document}
